\documentclass[a4paper, 11pt]{article}

\usepackage[justification=centering]{caption}
\usepackage[catalan]{babel}
\usepackage[utf8]{inputenc}
\usepackage{mathtools}
\usepackage{microtype}
\usepackage{pythontex}
\usepackage{amsfonts}
\usepackage{enumitem}
\usepackage{fancyhdr}
\usepackage{geometry}
\usepackage{hyperref}
\usepackage{graphicx}
\usepackage{listings}
\usepackage{verbatim}
\usepackage{amsmath}
\usepackage{float}
\usepackage[table,xcdraw]{xcolor}

\definecolor{codegreen}{rgb}{0,0.6,0}
\definecolor{codegray}{rgb}{0.5,0.5,0.5}
\definecolor{codepurple}{rgb}{0.58,0,0.82}
\definecolor{backcolour}{rgb}{0.95,0.95,0.92}

\lstdefinestyle{mystyle}{
    backgroundcolor=\color{backcolour},
    commentstyle=\color{codegreen},
    keywordstyle=\color{magenta},
    numberstyle=\tiny\color{codegray},
    stringstyle=\color{codepurple},
    basicstyle=\ttfamily\footnotesize,
    breakatwhitespace=false,
    breaklines=true,
    captionpos=b,
    keepspaces=true,
    numbers=left,
    numbersep=5pt,
    showspaces=false,
    showstringspaces=false,
    showtabs=false,
    tabsize=2
}

\lstset{style=mystyle}

\DisableLigatures{encoding = *, family = *}

\geometry{left=25mm, right=25mm, top=25mm, bottom=25mm}
\pagestyle{fancy}
\fancyhf{}
\lhead{Joel, Jordi, i Oriol}
\rhead{Aprenentatge Computacional}
\cfoot{\thepage}

\renewcommand{\headrulewidth}{0.6pt}
\renewcommand{\footrulewidth}{0.6pt}
\renewcommand{\baselinestretch}{1.5}
\setlength{\headheight}{13.6pt}

\title{\Huge{\textbf{Pràctica 2: Classificació}}}
\author{\Large{Joel Guevara Lopez, 1564581}
        \\\Large{Jordi Morales Casas, 1564921}
        \\\Large{Oriol Benítez Bravo, 1566931}}
\date{Aprenentatge Computacional, III MatCAD\\ \vspace{6pt} Desembre 2021}


\begin{document}

    \maketitle
    
    \section{Models de classificació aplicats al \textit{dataset} IRIS}
    
    La nostra base de dades consisteix en un conjunt de atributs que ens donen 
    les característiques de una flor i ens diu a que especie de flor pertany.
    Les variables són \textit{Sepal Length}, \textit{Sepal Width},  \textit{Petal
    Length} i \textit{Petal Width} i les especies a clasificar seràn \textit{Setosa}, 
    \textit{Versicolour} i \textit{Virginica}.
    
    En aquesta practica el que farem serà utilitzar diferents tipus de models de 
    classificació per poder trobar una predicció que ens decideixi la especie
    de la flor a partir de les seves característiques.
    \subsection{Models de classificació}
        \begin{itemize}
            \item \textbf{Regressió logística:}
                Observem com la classe 0 (Setosa) es classifica perfectament, la qual cosa indica que les seves característiques la diferencien de la resta. No succeeix el mateix per la resta de classes. A la corba PR veiem com no és possible assolir un \textit{recall} alt sense perdre molta precisió. Tot i això, si ens fixem en l'àrea sota la corba ROC, podem concloure que és bastant probable que distingeixi positivament aquestes classes. 
                
                \begin{figure}[H]%
                \centering
                \subfloat{{\includegraphics[width=7.5cm]{problr pr.png}}}%
                \qquad
                \subfloat{{\includegraphics[width=7.5cm]{problr rog.png}}}%
                \end{figure}
                
            \item \textbf{Regressió logística:}
                Observem com la classe 0 (Setosa) es classifica perfectament, la qual cosa indica que les seves característiques la diferencien de la resta. No succeeix el mateix per la resta de classes. A la corba PR veiem com no és possible assolir un \textit{recall} alt sense perdre molta precisió. Tot i això, si ens fixem en l'àrea sota la corba ROC, podem concloure que és bastant probable que distingeixi positivament aquestes classes. 
                
                \begin{figure}[H]%
                \centering
                \subfloat{{\includegraphics[width=7.5cm]{problr pr.png}}}%
                \qquad
                \subfloat{{\includegraphics[width=7.5cm]{problr rog.png}}}%
                \end{figure}

            
        \end{itemize}
\end{document}

